\documentclass[Main.tex]{subfiles}

\begin{document}

\title{\textit{\large{Appunti del corso di} \\[1ex] \Huge{Fisica Matematica}}}
\date{Anno accademico 2022-23}
\author{\Large{Pietro Scapolo}}



\clearpage
%\newgeometry{left=71pt, right=71pt, top=71pt, bottom=71pt, includefoot, headheight=13.6pt}

% temporary titles
% command to provide stretchy vertical space in proportion
\newcommand\nbvspace[1][3]{\vspace*{\stretch{#1}}}
% allow some slack to avoid under/overfull boxes
\newcommand\nbstretchyspace{\spaceskip0.5em plus 0.25em minus 0.25em}
% To improve spacing on titlepages
\newcommand{\nbtitlestretch}{\spaceskip0.6em}
\pagestyle{empty}
\begin{center}
\bfseries
\nbvspace[0.5]
\Huge
{
\textit{\Large{Appunti del corso di} \\[1ex] \Huge{Fisica Matematica}}}




\nbvspace[3]
{\fontsize{50}{60}\selectfont $\N \  \L \ \H$}
%\includegraphics[width=1.5in]{./graphics/pic37}
\nbvspace[3]
\normalsize

\nbvspace[1]

\Large Pietro Scapolo\\[0.5em]
\footnotesize Anno accademico 2022-2023
\end{center}

\restoregeometry
\newpage%per coerenza di stampa
\ \pagestyle{empty}
\newpage


%\thispagestyle{Prefazione}
\pagenumbering{Roman}
\pagestyle{empty}

\begin{center}
\addcontentsline{toc}{chapter}{Prefazione}
\begin{LARGE}{\textbf{Prefazione}}\end{LARGE} \bigskip

Questo documento contiene una raccolta di appunti relativi al corso di Fisica Matematica, tenuto dal professor A. Ponno presso il Dipartimento di Fisica e Astronomia "Galileo Galilei" dell'Università degli Studi di Padova durante l'anno accademico 2022-2023. Gli appunti, originariamente scritti in LaTeX da me, sono stati arricchiti integrandoli con le dispense dell'anno precedente, le quali sono state create in collaborazione con Manjodh Singh e Veronica Bedin. 

Desidero sottolineare che questi appunti rappresentano una rielaborazione personale e non sono stati revisionati dal professor A. Ponno. Pertanto, potrebbero contenere errori, e il professore non ha alcuna responsabilità sul loro contenuto. Questo documento non intende sostituire le lezioni del professore o altri appunti scritti da altri studenti, ma piuttosto offrire un'ulteriore prospettiva utile per comprendere i concetti trattati nel corso. Pertanto, consiglio prima di tutto di seguire le lezioni, e magari consultare questo documento e altri appunti per un ulteriore punto di vista.
\newline

Se trovate eventuali correzioni o suggerimenti segnalatemeli via mail!

\href{mailto:pietro.scapolo@studenti.unipd.it}{\underline{ pietro.scapolo@studenti.unipd.it}}

\bigskip
\begin{LARGE}{\textbf{Ringraziamenti}}\end{LARGE} \bigskip

Ringrazio inoltre, per il fantastico lavoro di revisione, i gentilissimi: Martina Tesser, Benedetta Rasera, Caterina Spadetto, Veronica Marchiori, Stefano Sorace Distilo e Pietro Pecchini. È inutile dire che senza il loro aiuto il testo non sarebbe com’è adesso e per questo siamo loro grati.
\end{center}


\indice
\thispagestyle{empty}

\end{document}








