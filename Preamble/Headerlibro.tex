% DIZIONARIO E FONT
\usepackage[T1]{fontenc}
\usepackage[utf8]{inputenc}
\usepackage[italian]{babel}
\usepackage{lmodern}
\usepackage[12pt]{moresize}

% GEOMETRIA
\usepackage{geometry}
%\usepackage[cam,a4,center]{crop}
%\geometry{left=71pt, right=71pt, top=71pt, bottom=71pt, includefoot, headheight=13.6pt, inner=120pt}

\newcommand{\Book}{\geometry{layoutwidth=6in, layoutheight=9in, 
            inner=1in, outer=0.75in, 
            top=0.75in, bottom=0.75in, 
            includefoot, headheight=13.6pt,
            bindingoffset=0.5in, showcrop}
            }


\newcommand{\setBookLayout}{
    \geometry{paper=a4paper, 
            layoutwidth=6in, layoutheight=9in, 
            layoutvoffset=1.345in, layouthoffset=1.135in,
            showcrop,
            left=1in, right=1in, top=1in, bottom=1in}
    \crop
}

% HYPERLINKS
\usepackage{hyperref}
\hypersetup{hidelinks, linktoc=all}

% LINGUAGGIO MATEMATICO
\usepackage{amssymb, amsmath, latexsym, mathrsfs}
\usepackage{physics}
\usepackage{bbold}
\usepackage{relsize}
\usepackage[makeroom]{cancel}

% APPENDICE
\usepackage{chngcntr}

\newcounter{appsection}[chapter]
\renewcommand{\theappsection}{\Alph{appsection}}

\newcommand{\appendice}[1]{
    \refstepcounter{appsection}
	\renewcommand{\thesubsection} {$\text{\theappsection}_\text{\Roman{section}}$}
	\subsection{#1}}



% GRAFICA
\usepackage{graphicx}
\usepackage{mdframed}
\usepackage{float}
\usepackage{wrapfig}
\usepackage{afterpage}
\graphicspath{images/}

% COMANDI DI STRUTTURA
\usepackage{caption}
\usepackage{xcolor}
\usepackage{enumitem}
\usepackage{changepage}

% INDICI E TITOLI
\usepackage{etoolbox}
\usepackage{titlesec}
\usepackage{tocloft}
\usepackage{etoc}

% TITOLO
\makeatletter         
\renewcommand\maketitle{
{\raggedright 
\begin{center}
{\Huge \textbf{ \@title }}\\[4ex] 
{\Large  \@author}\\[2ex] 
\large \@date\\[8ex]
\end{center}
\thispagestyle{empty}
\newpage \pagenumbering{arabic}
}} \makeatother

%\titleformat{\chapter}{\Huge\bfseries}{}{0em}{}
\titleformat{\section}
{\LARGE\bfseries}{\thesection}{1em}{}
\titleformat{\subsection}
{\large \bfseries}{\thesubsection}{1em}{}
\renewcommand{\cftchapfont}{\large \bfseries }



%INDICI
\addto\captionsitalian{\renewcommand{\contentsname}{\Huge Contenuti}}

\newcommand{\toc}{\etocsettocstyle{\vskip-2\baselineskip\noindent\rule{\linewidth}{0.5pt}\vskip0.3\baselineskip}{\noindent\rule{\linewidth}{0.5pt}\vskip0.5\baselineskip} \vspace{28.4pt} \setcounter{tocdepth}{3}\localtableofcontents}

\newcommand{\indice}{
\clearpage
\pagestyle{empty}
\setcounter{tocdepth}{3}
\renewcommand{\cftsecfont}{\bfseries}
\etocsettocstyle{\vskip-2\baselineskip\noindent\vskip0.3\baselineskip}{\noindent\vskip0.5\baselineskip} \tableofcontents}




% ALTRI PACCHETTI
\usepackage{comment}
\usepackage{marginnote}
\usepackage{footnotehyper}
\usepackage[bottom]{footmisc}
\renewcommand*{\marginfont}{\footnotesize\itshape}



% TEOREMI
\usepackage{amsthm}
\newtheoremstyle{a capo}% nome dello stile
  {\topsep}% spazio sopra
  {\topsep}% spazio sotto
  {\itshape}% font del corpo
  {}% indentazione del primo paragrafo
  {\bfseries}% font dell'intestazione
  {}% punteggiatura dopo l'intestazione
  {\newline}% spazio dopo l'intestazione
  {\thmname{#1}\thmnumber{ #2}\thmnote{ (#3)}}% formato dell'intestazione
 \newtheoremstyle{due punti}% nome dello stile
  {\topsep}% spazio sopra
  {\topsep}% spazio sotto
  {\itshape}% font del corpo
  {}% indentazione del primo paragrafo
  {\bfseries}% font dell'intestazione
  {:}% punteggiatura dopo l'intestazione
  { }% spazio dopo l'intestazione
  {\thmname{#1}\thmnumber{ #2}\thmnote{ (#3)}}% formato dell'intestazione
  
\newtheorem{pro}{Proposizione}
\newtheorem{cor}{Corollario}

\theoremstyle{due punti}
\newtheorem{df}{Definizione}
\theoremstyle{a capo}
\newtheorem{teo}{Teorema}
\newtheorem{pr}{Principio}
\newcommand\oss{\noindent \textbf{Osservazione: }}




% CONTATORI
\usepackage{chngcntr}
\usepackage{alphalph}
\renewcommand{\thesection}{\arabic{section}}
\counterwithin{df}{section}
\counterwithin*{pr}{section}
\counterwithin{pro}{section}
\counterwithin{cor}{section}
\counterwithin{teo}{section}
\newcounter{es}
\counterwithin*{es}{chapter}
\newcounter{ese}
\counterwithin*{ese}{section}
\counterwithin{equation}{section}


% DERIVATE E COMANDI MATEMATICI
\newenvironment{pp}[2]{\frac{\partial #1}{\partial #2}}{}
\renewenvironment{dd}[2]{\frac{d #1}{d #2}}{}
\renewcommand{\parallel}{\mathrel{/\mkern-5mu/}}
\makeatletter
\newcommand{\notparallel}{
  \mathrel{\mathpalette\not@parallel\relax}}
\newcommand{\not@parallel}[2]{
  \ooalign{\reflectbox{$\m@th#1\smallsetminus$}\cr\hfil$\m@th#1\parallel$\cr}}
\newcommand{\longmatrix}[4]{\left( \begin{matrix}     #1 & \dots  & #2\\ \vdots & \ddots & \vdots\\ #3 & \dots  & #4 \end{matrix} \right)}
\renewcommand{\vec}[1]{\left( \begin{array}{c} #1 \end{array} \right)}
\def\rin{\rotatebox[origin=c]{90}{$\in$}}



% AMBIENTI
%IMMAGINE
\newenvironment{immagine}[2]{ x}{}

%TEMA
\definecolor{tema}{RGB}{189,92,04}
\newcommand{\tem}{A}
\newcounter{j}
\setcounter{j}{0}
\newcounter{savedequation}
\renewcommand{\thees}{\tem.\textit{\arabic{es}}} % Notazione personalizzata
\newenvironment{tema}[1][]{
  \stepcounter{j} 
  \medskip 
  \refstepcounter{es}%
  \begin{mdframed}[linewidth=1.5,linecolor=tema, topline=false,rightline=false, bottomline=false]  
    \addcontentsline{toc}{subsubsection}{\underline{\thees} \hspace{3pt} #1}
    \textcolor{tema}{\textbf{\large Tema~(\thees){: #1} \newline}}
    \setcounter{savedequation}{\value{equation}} 
    \renewcommand{\theequation}{\thees~- \arabic{equation}}
    \setcounter{equation}{0}
    \label{es:\thej}\noindent}{
  \setcounter{equation}{\value{savedequation}}
  \end{mdframed}
}


\newcommand{\theap}{\Roman{section}}
 % Notazione personalizzata
\newenvironment{appendic}[1][]{
  \stepcounter{j} 
  \medskip 
  \refstepcounter{es}%
  \begin{mdframed}[linewidth=1.5,linecolor=teal, topline=false,rightline=false, bottomline=false]  
    \textcolor{teal}{\textbf{\large #1} \newline}
    \setcounter{savedequation}{\value{equation}} 
    \renewcommand{\theequation}{\theap. \arabic{equation}}
    \setcounter{equation}{0}}{
  \setcounter{equation}{\value{savedequation}}
  \end{mdframed}}




%ESERCIZIO
\newenvironment{esercizio}[1][]{\medskip \stepcounter{ese}\noindent \fbox{\textbf{Esercizio~\thesubsection.\theese}}\hrulefill\par\vspace{10pt}\noindent\rmfamily}{\par\noindent\hrulefill\vrule width10pt height2pt depth2pt\par}
%DIMOSTRAZIONE
\newenvironment{dm}[1][] {\medskip \begin{mdframed}[linewidth=1.5,linecolor=gray, topline=false,rightline=false,bottomline=false] \textbf{Dimostrazione:}{ \itshape #1\ } \newline}{\medskip \hfill{$\square$} \end{mdframed}\bigskip}
%ESEMPIO
\definecolor{gialloverde}{RGB}{0,125,9}
\newenvironment{esempio}[1][] {\medskip  \begin{mdframed}[linewidth=1.5,linecolor=gialloverde, topline=false,rightline=false,bottomline=false] \textcolor{gialloverde}{\textbf{Esempio{: #1} \newline}}}{\medskip\end{mdframed} }

%OSSERVAZIONE
\newlist{osservazioneenum}{enumerate}{1}
\setlist[osservazioneenum]{label=(\textit{\roman*})\hspace{0.5em}, leftmargin=0pt, labelsep=0pt, itemindent=2em}

\newenvironment{osservazioni}[1][]{
  \medskip
  \begin{adjustwidth}{21.88pt}{}
    \textbf{\textit{Osservazioni:}}\newline
    \vspace{-14.2pt}
    \begin{osservazioneenum}[label=(\textit{\roman*})\hspace{0.5em}, leftmargin=0pt, labelsep=0pt, itemindent=2em]}{
    \end{osservazioneenum}
  \end{adjustwidth}}
  
\newenvironment{osservazione}[1][]{
  \bigskip
  \begin{adjustwidth}{21.88pt}{}
    \textbf{\textit{Osservazione:}}\vspace{2.84pt} \newline}{
  \everypar{\leftskip=0pt} % Ripristina il rientro del testo
  \end{adjustwidth} \bigskip }




% IMPAGINAZIONE
\usepackage{fancyhdr}
\usepackage{ifthen}
\usepackage{fancyhdr}% http://ctan.org/pkg/fancyhdr
\fancypagestyle{dispensa}{%
\pagenumbering{arabic}
\fancyhf{}
\renewcommand{\headrulewidth}{0.4pt}
\fancyhead[OR]{\textbf{\leftmark}}
\fancyhead[EL]{\textbf{\rightmark}}
\fancyfoot[LE,RO]{\thepage}}

\fancypagestyle{Prefazione}{%
\fancyhf{}
\renewcommand{\headrulewidth}{0pt}
\fancyfoot[C]{\thepage}}

\usepackage{ifthen}

\makeatletter
\renewcommand{\chapter}{%
  \checkoddpage
  \ifoddpage
    \newpage
    \thispagestyle{empty}
    \null
    \newpage
  \fi
  \global\@topnum\z@
  \@afterindentfalse
  \secdef\@chapter\@schapter
}
\makeatother

\let\originalchapter\chapter
\renewcommand{\chapter}[2][]{%
  \checkoddpage
  \ifoddpage
    \newpage
    \thispagestyle{empty}
    \null
    \newpage
  \fi
  \ifthenelse{\equal{#1}{}}%
    {\stepcounter{chapter} \originalchapter*{#2}%
     \addcontentsline{toc}{chapter}{#2}}%
    { %
     \addcontentsline{toc}{chapter}{#1}}%
       \ifstrempty{#1}
    {\markboth{\MakeUppercase{#2}}{\rightmark}}%
    {\markboth{#1}{\rightmark}}%
}

\let\originalsection\section
\renewcommand{\section}[2][]{%
  \ifstrempty{#1}
    {\originalsection{#2}
     \markright{\thesection. #2}}%
    {\originalsection[#1]{#2}
     \markright{\thesection. #1}}%
}


%ALFABETO MATEMATICO
\newcommand{\A}{\mathcal{A}}
\newcommand{\B}{\mathcal{B}}
\newcommand{\C}{\mathcal{C}}
\newcommand{\Cc}{\mathbb{C}}
\newcommand{\D}{\mathcal{D}}
\newcommand{\E}{\mathcal{E}}
\newcommand{\F}{\mathcal{F}}
\newcommand{\G}{\mathcal{G}}
\renewcommand{\H}{\mathscr{H}}
\newcommand{\I}{\mathcal{I}}
\newcommand{\K}{\mathcal{K}}
\renewcommand{\L}{\mathscr{L}}
\newcommand{\M}{\mathcal{M}}
\newcommand{\N}{\mathscr{N}}
\renewcommand{\O}{\mathcal{O}}
\renewcommand{\P}{\mathcal{P}}
\newcommand{\Q}{\mathcal{Q}}
\newcommand{\Qq}{\mathbb{Q}}
\newcommand{\R}{\mathcal{R}}
\newcommand{\Rr}{\mathbb{R}}
\renewcommand{\S}{\mathcal{S}}
\newcommand{\T}{\mathcal{T}}
\newcommand{\U}{\mathcal{U}}
\newcommand{\V}{\mathcal{V}}
\newcommand{\W}{\mathcal{W}}
\newcommand{\X}{\mathcal{X}}
\newcommand{\Y}{\mathcal{Y}}
\newcommand{\Z}{\mathcal{Z}}
\newcommand{\1}{\mathbb{1}}