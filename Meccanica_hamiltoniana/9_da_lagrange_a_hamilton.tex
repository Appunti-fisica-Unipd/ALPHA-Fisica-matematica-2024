\documentclass[Main.tex]{subfiles}

\begin{document}
\section{Formalismo Hamiltoniano}
\toc

\subsection{Esplicitare la variabile velocità}
L'idea è quella di considerare il momento Lagrangiano coniugato: $$p = \pp{\L}{\dot q}$$. Ci si chiede in quali casi è possibile invertire questa equazione e scrivere la $p$ come variabile indipendente. Si sa dall'analisi che questo è possibile localmente $\iff$ il Jacobiano della funzione è non singolare, ovvero deve valere: 
\begin{equation}
\mathrm{det}(\frac{\partial^2 \L}{{\partial \dot q}^2}) \neq 0    
\end{equation}
In tal caso la funzione inversa si indicherà 
\begin{equation}
\dot q = f(q,p,t)    
\end{equation}
\begin{osservazione}
    Per i sistemi meccanici Lagrangiani questo vale sempre. Infatti in generale $$\L = K_2 + K_1 + K_0 - U$$ con la solita notazione. Allora per verificare la condizione è sufficiente considerare il termine $K_2$ in quanto, essendo quadratico nelle $\dot q$, è l'unico che sopravvive alla doppia derivazione. Allora:
    \begin{gather}
        K_2 = \frac{1}{2} \dot q \cdot A(q,t) \dot q \\ \implies
        \frac{\partial^2 K_2}{{\partial \dot q}^2} = \frac{\partial^2 \L}{{\partial \dot q}^2} = A(q,t)
    \end{gather}
    Si sa che la matrice cinetica $A(q,t)$ è definita positiva e quindi in particolare è non singolare $\forall (q,t) \in (R^N, R)$.
    Allora si avrà:
    \begin{gather}
        p = A(q,t)\dot q + \alpha(q,t) \implies \dot q = f(q,p,t) = A^{-1}(q,t)\big( p - \alpha(q,t) \big)
    \end{gather}
\end{osservazione}
\begin{pro}
    Sia $\L(q,\dot q, t)$ Lagrangiana tale che $\frac{\partial^2 \L}{{\partial \dot q}^2}$ è non singolare e sia $$H(q,p,t) \coloneq \mathcal{H}(q,f(q,p,t), t) = \big( \pp{\L}{\dot q} \cdot \dot q - \L \big) \big\rvert_{\dot q = f(q,p,t)}$$
    Dove $\mathcal{H}$ è la funzione energia Lagrangiana. Allora le equazioni di Lagrange sono equivalenti a quelle di Hamilton, ovvero 
    \begin{gather}
        \dd{}{t}\pp{\L}{\dot q} = \pp {\L}{q} \iff  \\ \begin{cases} 
          \pp {H}{p} = \dot q \\
          \pp {H}{q} = - \dot p\\
       \end{cases}
    \end{gather}
\end{pro}
\begin{dm}
    Si procede calcolando il differenziale di $H(q,p,t) = p \cdot \dot q - \L(q,\dot q, t) \rvert_{\dot q = f(q,p,t)}$. Questo è
    \begin{gather}
        dH = dp \cdot f + p \cdot df - \pp{\L}{q} \cdot dq - \pp{\L}{f} \cdot df - \pp{\L}{t} dt = \\ -\pp {\L}{q} \cdot dq + \big( p - \pp{L}{f} \big)\cdot df + f \cdot dp - \pp{\L}{t}dt
    \end{gather}
    D'altra parte essendo $H = H(q,p,t)$ si ha che
    \begin{equation}
        dH = \pp{H}{q}\cdot dq + \pp{H}{p}\cdot dp + \pp{H}{t}dt
    \end{equation}
    Allora vale
    \begin{equation}
        \begin{cases}
            \pp{H}{q} = - \pp{L}{q}\\
            \pp{H}{p} = f(q,p,t) = \dot q\\ 
            \pp{H}{t} = -\pp{\L}{t}
        \end{cases}
    \end{equation}
    Allora se vale Lagrange: $\pp{\L}{q} = \dot p$ e un'implicazione è dimostrata. Viceversa ricordando che 
    \begin{gather}
        \dot{\mathcal{H}} + \pp{\L}{t} = 0 \implies \dot H + \pp{\L}{t} = 0
    \end{gather}
    e calcolando:
    \begin{gather}
        \dot H = \dd{}{t} H(q,p,t) = \pp{H}{q} \cdot \dot q + \pp{H}{p} \cdot \dot p + \pp{H}{t}
    \end{gather}
    Usando allora le equazioni di Hamilton resta: $$\dot H = \pp{H}{t} = -\pp{\L}{t}$$ 
\end{dm}




\subsection{Forma della lagrangiana}
Nel caso n-dimensionale la lagrangiana assume la forma
\begin{gather}
	\dd{}{t} \pp{\L}{\dot q_n} = \pp{\L}{q_n} \ \ \ \ (n=1,...L)\\
	\iff \dd{}{t} \pp{\K}{\dot q_n} = \pp{\L}{q_n} 
\end{gather}
Il membro di sinistra è dato da
\begin{gather}
	\dd{}{t} \left[ \pp{\K}{\dot q_n} \right]= \dd{}{t} \left[\pp{\K_2}{\dot q_n} + \pp{\K_1}{\dot q_n}\right]=\\
	=\dd{}{t} \left[\sum_{\ell=1}^L A_{n \ell} \dot q_\ell + B_n\right]=\\
	= \sum_{\ell=1}^L \left( \sum_{m=1}^L \pp{A_{n \ell}}{ q_m} \dot q_m + \pp{A_{n\ell}}{t} \right) \dot q_\ell + \sum_{\ell=1}^L A_{n \ell} \ddot q_\ell + \sum_{\ell=1}^L \pp{B_n}{q_\ell}  \dot q_\ell + \pp{B_n}{t} \notag
\end{gather}
Siccome sono presenti sia la forma quadratica, che lineare che di grado 0, si può riordinare per ordini
\begin{gather}
	\sum_{\ell=1}^L A_{n \ell} \ddot q_\ell + \sum_{\ell,m=1}^L   \pp{A_{n \ell}}{ q_m} \dot q_m \dot q_\ell + \sum_{\ell=1}^L \left(\pp{B_n}{q_\ell}+\pp{A_{n\ell}}{t} \right) \dot q_\ell + \pp{B_n}{t}
\end{gather}
Il mebro di destra è dato da
\begin{gather}
	\pp{\L}{q_n} = \pp{\K_2}{q_n} + \pp{\K_1}{q_n} + \pp{\K_0}{q_n} - \pp{\U}{q_n}=\\
	=\frac{1}{2} \sum_{\ell, m=1}^L  \pp{A_{\ell,m}}{q_n} \dot q_\ell \dot q_m + \sum _{\ell=1}^L \pp{B_\ell}{q_n} \dot q_\ell + \pp{\K_0}{q_n} - \pp{\U}{q_n}
\end{gather}
Nell'equazione totale, portando tutto a destra si ha 
\begin{gather}
	\sum_{\ell=1}^L A_{n\ell} \ddot q _\ell = - \sum _{\ell,m=1}^L  \underbrace{\left( \pp{A_{n \ell}}{q_m} - \frac{1}{2} \pp{A_{\ell m }}{q_n} \right)}_{:=\gamma_{n \ell}^n} \dot q_\ell \dot q_m +\\+ \sum_{\ell=1}^L \left( \underbrace{\pp{B_\ell}{q_n} - \pp{B_n}{q_\ell}}_{:=G_{n\ell}} - \pp{A_{n\ell}}{t} \right) \dot q_\ell - \pp{B_n}{t} - \pp{(\U- \K_0)}{q_n} 
\end{gather}
A vista si vede che la parte $\pp{B_\ell}{q_n} - \pp{B_n}{q_\ell}$ è antisimmetrica (scambiando gli indici si ottiene un'uguaglianza con il meno), $G$ sarebbe quindi lo jacobiano parziale di $B$, che è un campo di vettori, e quando si traspone si ottiene appunto $-G$, è quindi una matrice antisimmetrica che moltiplica la velocità. È della forma \emph{geoscopica} ed è il motivo per cui $\K_1=0$. Scrivendo l'espressione per l'n-esima equazione di Lagrange si vede che
\begin{gather}
	\sum_{\ell=1}^L A_{n \ell} \ddot q_\ell = - \sum_{\ell, m =1}^L \gamma_{\ell,m}^n \dot q_\ell \dot q_m + \sum_{\ell=1}^L G_{n \ell} \dot q_\ell + \\ - \sum_{\ell=1}^L \pp{A_{n \ell}}{t} \dot q_\ell - \pp{B_n}{t} - \pp{(U-\K_0)}{t}
\end{gather}
Dove $G_{n \ell} =\pp{B_\ell}{q_n} - \pp{B_n}{q_\ell}$ e $\gamma_{\ell m}^n = \left( \pp{A_{n \ell}}{q_m} - \frac{1}{2} \pp{A_{\ell m}}{q_m} \right)$, oppure in maniera più simmetrica si può scrivere:
\begin{gather}
	\gamma_n (\dot q) = \sum_{\ell, m =1}^L \left( \pp{A_{n \ell}}{q_m} - \frac{1}{2} \pp{A_{\ell m}}{q_m} \right) \dot q_\ell \dot q_m =\\= \frac{1}{2} \sum_{\ell , m =1 }^L \left( \pp{A_{n \ell}}{q_m} + \pp{A_{n m}}{q_\ell} - \pp{A_{\ell m}}{q_n} \right) \dot q_\ell \dot q_m
	\end{gather}

Passando ora  alla forma vettoriale si ha
\begin{gather}
	A \ddot q = - \gamma (\dot q) + G \dot q - \pp{A}{t} \dot q - \pp{B}{t} - \pp{(\U- \K_0)}{q}
\end{gather}
Siccome è $A$ è una matrice definita positiva si può scrivere
\begin{gather}
	\ddot q = A^{-1} \left[ - \gamma (\dot q) + G \dot q - \pp{A}{t} \dot q - \pp{B}{t} - \pp{(\U- \K_0)}{q} \right]
\end{gather}
Allora dalle equazione di Lagrange è esprimibile l'accelerazione e si possono appplicare tutti i teoremi noti delle equazioni differenziali ordinarie e si potrebbero anche riscrivere come un sistema del primo ordine con la solita sotituzione nello spazio delle fasi.

\begin{osservazione}
Se $\pp{X}{t} =0 $ e $\pp{U}{t}=0$ allora $
\ddot q = A^{-1} \left[- \gamma (\dot q) - \pp{\U}{q} \right]	$. In questo caso la lagrangiana di partenza è
\begin{equation}
	\L = \frac{1}{2} \dot q \cdot A(q) \dot q - \U (q, \dot q)
\end{equation}
Un tale sistema è detto sistema meccanico naturale e conservativo (perchè $q$ non dipende da $t$).
Per i moti geodetici (cioè \emph{liberi} da forze esterne) l'equazione è $\ddot q=A^{-1} \gamma (\dot q)$ e allora $\L=\K_2$.
\end{osservazione}

\subsection{Geodetiche}
Come osservato l'equazione della lagrangiana per le geodetiche può essere espressa in forma
\begin{equation}
	\ddot q=A^{-1} \gamma (\dot q)
\end{equation}
Per componenti si ha
\begin{gather}
	\ddot q_k = \left( A^{-1} (q) \gamma ( \dot q)\right)_k = \sum_{n=1}^L (A^{-1})_{kn} \gamma_n (\dot q) = \\
	= - \sum_{\ell ,m =1}^L 	\underbrace{\left[ \sum_{n=1}^L (A^{-1}(q))_{kn} \frac{1}{2}  \left( \pp{A_{n \ell}}{q_m} + \pp{A_{n m}}{q_\ell} - \pp{A_{\ell m}}{q_n} \right)\right]}_{\Gamma_{\ell m}^k (q)} \dot q_\ell \dot q_m=\\
	\Rightarrow \boxed{ \ddot q_k = -\sum_{\ell, m=1}^L \Gamma_{\ell m}^k \dot q_\ell \dot q_m }
\end{gather}
Il coeffieciente $\Gamma ^k_{\ell m}$ è detto simbolo di Christoffel o simboli di connessione associati ad $A(q)$.

\end{document}